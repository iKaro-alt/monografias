\documentclass[article, 11pt, oneside, a4paper, english, brazil, sumario=tradicional]{abntex2}
\usepackage{lmodern}
\usepackage[T1]{fontenc}
\usepackage[utf8]{inputenc}
\usepackage{indentfirst}
\usepackage{nomencl}
\usepackage{color}
\usepackage{graphicx}
\usepackage{microtype}
\usepackage{lipsum}
\usepackage[brazilian,hyperpageref]{backref}
\usepackage[alf]{abntex2cite}
\renewcommand{\backrefpagesname}{Citado na(s) página(s):~}
\renewcommand{\backref}{}
\renewcommand*{\backrefalt}[4]{
	\ifcase #1 %
		Nenhuma citação no texto.%
	\or
		Citado na página #2.%
	\else
		Citado #1 vezes nas páginas #2.%
	\fi}%
% --- FOLHA DE ROSTO ---
\titulo{A importância do ensino de tecnologia da computação nas escolas}
\tituloestrangeiro{The importance of the teaching of computation technology in schools}
\autor{
Equipe \abnTeX\thanks{``Recomenda-se que os dados de vinculação e
endereço constem em nota, com sistema de chamada próprio, diferente do sistema
adotado para citações no texto.'' \url{http://www.abntex.net.br/}}
\\[0.5cm]
Ícaro Onofre Silva, Daniel Antunes Vieira, Maria Isabel\thanks{``Constar currículo sucinto de cada autor, com
vinculação corporativa e endereço de contato.''}}
\local{Brasil}
\data{2022}
\definecolor{blue}{RGB}{41,5,195} % alterando o aspecto da cor azul
\makeatletter % informações do PDF
\hypersetup{
		pdftitle={\@title},
		pdfauthor={\@author},
    	pdfsubject={Modelo de artigo científico com abnTeX2},
	    pdfcreator={LaTeX with abnTeX2},
		pdfkeywords={abnt}{latex}{abntex}{abntex2}{atigo científico},
		colorlinks=true,
    	linkcolor=blue,
    	citecolor=blue,
    	filecolor=magenta,
		urlcolor=blue,
		bookmarksdepth=4
}
\makeatother

\makeindex

\setlrmarginsandblock{3cm}{3cm}{*}
\setulmarginsandblock{3cm}{3cm}{*}
\checkandfixthelayout
\setlength{\parindent}{1.3cm}
\setlength{\parskip}{0.2cm}
\SingleSpacing

\begin{document}

\selectlanguage{brazil}

\frenchspacing

% \twocolumn[    		% Descomente para duas colunas
\maketitle
% resumo em português
\begin{resumoumacoluna}
  Aqui vai o resumo
 \vspace{\onelineskip}
 \noindent
 \textbf{Palavras-chave}: latex. abntex. editoração de texto.
\end{resumoumacoluna}

% resumo em inglês
\renewcommand{\resumoname}{Abstract}
\begin{resumoumacoluna}
 \begin{otherlanguage*}{english}
   According to ABNT NBR 6022:2018, an abstract in foreign language is optional.

   \vspace{\onelineskip}

   \noindent
   \textbf{Keywords}: latex. abntex.
 \end{otherlanguage*}
\end{resumoumacoluna}
% ]  				% FIM DE ARTIGO EM DUAS COLUNAS
\begin{center}\smaller
\textbf{Data de submissão e aprovação}: elemento obrigatório. Indicar dia, mês e ano

\textbf{Identificação e disponibilidade}: elemento opcional. Pode ser indicado
o endereço eletrônico, DOI, suportes e outras informações relativas ao acesso.
\end{center}

\pdfbookmark[0]{\contentsname}{toc}
\tableofcontents*
\cleardoublepage

\textual

\section{Introdução}
    A importância de se tornar um cidadão digital hoje em dia é de mesma importância
que ser alfabetizado, e para este processo de emancipação é necessário fazer uso
de técnicas de ensino que utilizem tecnologia, para que o aluno consiga assimilar
os inúmeros conceitos envolvidos nos processos de computação, introduzir os
alunos a estes conceitos fará que eles possam gradativamente assimilar a vasta
cultura da era da informação,  que em sua complexidade máxima ultrapassa a
possibilidade de se compreender totalmente, em vez disso, constrói-se uma
postura em relação ao conhecimento. Esta postura em relação ao conhecimento é o
que faz com que o indivíduo possa participar de forma ativa dos debates sobre
tecnologia, e possa buscar informações que sirvam de insumo no seu julgamento
crítico do mundo em que vive, a atual postura do cidadão brasileiro em relação a
tecnologia é passiva, onde ele tem a sua vida ditada pela tecnologia que não
compreende.
    Para mudar esta realidade e aprender computação e os conceitos abstratos
envolvidos na área atividades como o ensino de lógica de programação através
da internet e navegadores, ou o trabalho com a circuitaria encontrada em
dispositivos do dia a dia para criar brinquedos lúdicos, podem desenvolver aquilo
que [Wing 2006] define como pensamento computacional(PC), para [Blikstein 2008 o uso não
está limitado ao uso básico dos computadores, como escrever com editores de texto ou utilizar sistemas
de email.
    Neste trabalho exploraremos estes conceitos através da pesquisa da literatura
desenvolvida na área de educação.

\section{Contexto social do ensino de tecnologia}
    É preciso compreender o novo contexto social em que se vive para compreender
influenciadas pela tecnologia, um exemplo disso é a alfabetização de crianças,

onde a tecnologia da escrita e da comunicação através de símbolos, transformou a
vida humana profundamente.

\begin{citacao}
Na verdade, desde o inicio da civilização, o predomonio de um determinado tipo
de tecnologia transforma o comportamento pessoal e social de todo o grupo. Não é
por acaso que todas as eras foram, cada uma à sua maneira,"eras tecnológicas".
Assim tivemas a Idade da Pedra, do Bronze, até chegarmos ao momento tecnológico
atual, da Sociedade da Informação ou Sociedade Digital. (KENSKI,2003,p.48).
\end{citacao}

    Nesta era da informação não será diferente, novas estruturas sociais estão
surgindo por causa da tecnologia, e com isso novos impasses e dilemas sociais,
dilemas esses que muitos brasileiros não estão preparados ainda para lidar. Casos
recentes como: o incidente da Cambridge Analytica que ocorreu no periodo de 2014 até 2018 quando o caso começou a ser investigado, onde uma empresa privada fez
uso de algoritmos de mineração de dados e análise de dados para realizar
raspagem(scrapping), termo usado quando um algoritmo obtém dados através de
sites, para interferir na política dos Estados Unidos através de campanhas de
políticos. A Cambridge Analytica coletou de milhões de usuários
com o intuito de perfilar modelos psicológicos dos usuários com o intuito de
criar campanhas políticas mais direcionadas a certos públicos. O jornal The New
York Times informou que o senador Ted Cruz estava os dados deste sistema para
ganho eleitoral, com este caso ficou claro a possibilidade de empresas privadas
trabalharem em cima dos dados dos usuários de redes sociais, violando os valores
da privacidade, para interferir nos processos eleitorais de países.

    Harari em uma palestra que que concedeu à câmara dos deputados junto ao projeto modernizar,
que os dados se tornaram o novo petróleo, este conceito expressa mudança social provocada pela tecnologia,
e demonstra a importância do uso consciente dos programas para não se tornar a
mercadoria nos novos mercados emergentes. Já houve uma mudança na legislação brasileira
com a criação da lei geral de proteção de dados (lei n13.709/2018) em relação aos novos
cenários da era da informação, mas dados como a carência de mão de obra especializada na área
de tecnologia ainda demonstram, as mudanças ainda tardam a chegar.

    Para incorporar o ensino de tecnologia, é possível lecionar os tópicos de informática de maneira direta, ou indireta
ao incentivar o uso de ferramentas digitais para a solução de atividades executadas
pelo aluno, estudando diretamente os conceitos, os professores poderiam abordar
diretamente questões da atualidade, e incentivar projetos de pesquisa, um
exemplo de um conceito emergente na área da computação é o conceito de internet
da coisas, que é a integração de inúmeros sistemas embarcados à rede de computadores.
    Particularmente o conceito de internet das coisas permite ensinar
aspectos fundamentais do dia a dia que frequentemente ficam omissos no estudo
convencional atual de tecnologia da informação, como por exemplo, como funcionam
os sistemas embarcados de máquinas de cartão de crédito ou as urnas eletrônicas?
Inúmeras questões surgem todos os dias a respeito de computação que são
falsamente respondidas devido a falta de conhecimento na cultura brasileira,
hoje em dia computadores funcionam  na cabeça das pessoas como os criadores do
UNIX abstraiam os processos do sistema clássico, daemons, demônios ou entidades
por trás da máquina que agem de maneira fantasmagórica.  Outras implementações
no ensino de computação e os conceitos abstratos envolvidos na área, atividades
como o ensino de lógica de programação através da internet e navegadores, ou o
trabalho com a circuitaria encontrada em dispositivos do dia a dia para criar
brinquedos lúdicos, podem trazer esses daemons do mundo do obscurantismo para o
mundo da racionalidade.

    Incorporação do ensino da tecnologia no modelo tradicional não visa apenas suprir
as necessidades mercadológicas, e sim desenvovler um pensamento computacional, onde o indíviduo
consegue abstrair os processos computacionais através do pensamento computacional(PC).

    Para [Wing 2006] o conceito fundamental do PC é a abstração
de problemas em programas e processos computacionais, mas não se deve confundir a integração
do ensino da tecnologia ao atual ensino convencional como uma tentativa de criar uma geração
de mão de obra especializada, é a percepção que a capacidade do pensamento computacional se tornará
tão fundamental quanto a escrita.

    Ao introduzir estes conceitos no processo educacional, será possível não somente
suprir as necessidades do mercado de tecnologia, e sim criar uma geração de indivíduos
prontos para lidar com as questões sociais do futuro, é crível que a solução dos problemas
atuais da humanidade virá de uma profunda integração da computação em todas as áreas da atuação
humana, com a internet das coisas, o que se espera é uma quarta revolução industrial onde
sistemas embarcados de todas as dimensões integraram todos os processos, desde a saúde
básica até as mais complexas plantas industriais, onde ficará o ser humano neste novo cenário
de tecnologia? Quais seram os impasses da condição humana no futuro? É impossível dizer,
casos análogos ao da Cambridge Anlytica de dimensões desconhecidas aguardam a humanidade
num futuro incerto.

    Outro caso recente de como o descompasso da atual população com a tecnologia causou
crises na sociedade brasileira foi o caso do inquérito das fake news, onde foi avaliado
a criação de notícias falsas de maneira sistêmica com o intuito de influenciar as opiniões
da população e espalhar a desinformação. A falta de experiência da população com essas
estratégias midiáticas tornou-as efetivas, e habituar a população às tecnologias pode fazer que
ela consiga distinguir informação de ruído, as novas questões epistemológicas vindas do acesso ao excesso [Costa 2008]
de informação, também são desafios no novo ensino.

\begin{citacao}
O ensino superior está imerso em grandes desafios, tendo em vista as demandas e tendências desse novo contexto tecnológico. Estão em xeque a estrutura, o currículo, os  espaços,  os  tempos  e  os  modelos  de  ensino  e  de  aprendizagem  utilizados  até então,  bem como  os  papéis  desempenhados  por  docentes  e  estudantes  na  relação com o conhecimento socialmente válido. (KENSKI,2019,p.145).
\end{citacao}

\section{Como as tecnologias digitais de informação e comunicação afetam o ensino}
    As tecnologias digitais de informação e comunicação afetaram o ensino
de uma maneira inesperada, quando todo o sistema de ensino se tornou dependente dessas tecnologias
com a pandemia de COVID-19 atingiu o Brasil, ficou claro que o autodidatismo trazido pelo fácil
acesso à informação se tornou protagonista na vida dos estudantes que se encontraram em meio a crise
sanitária, essas novas condições questionavam a posição atual do professor e como as tecnologias poderiam ser usadas
para fomentar um ensino mais eficiente, o processo de questionar a atual estrutura de ensino nas escolas foi
catalizada pelas novas condições impostas pela pandemia.

    Os currículos escolares foram questionados pelos educandos devido à maneira
em que os mesmos obtêm informação, não se compreende a abordagem meramente
conteúdista mais como a ideal, pois é possível facilmente aceder conteúdos
diversos e profundos de maneira rápida.

\begin{citacao}
Levy  (1999)  pontua  que  vivenciamos  uma  mutação  na  relação  com  o
saber.  Os universitários  possuem  novos  interesses  e  habilidades,
sobretudo  nos  usos  dos recursos digitais. Não se interessamem permanecer em
aulas massivas –plenas do protagonismo  do  docente  e  a  passividade  dos
estudantes –cujos  objetivos  estão ligados  à  reprodução  de  conteúdos,  de
acordo  com  a  perspectiva  apresentada  pelo mestre.  Exigem  outros  modos
de  ensino,  mais  velozes e  participativos.  Por  meio  de seus equipamentos
digitais móveis acessam informações e interagem com o resto do mundo  em  busca
de  saberes  que  estão  disponíveis  em  qualquer  lugar.  Mídias inteligentes
conectadas   à   internet   que   não   conferem   somente   mobilidade e
convergência.  Oferecem  também  versatilidade  e  rapidez  em  interações
amplas,  no acesso a informações e em tomada de decisões em diferentes âmbitos
da vida, o que exige novas habilidades.(KENSKI,2019)
\end{citacao}

    Os métodos de ensino deveram ser aprimorados às novas realidades, a realidade do acesso
ao excesso da informação mudou a dinâmica que se enxerga o saber, a questão epistemológica
de como encarar o universo de informação que um individuo tem acesso mudou, ao adentrarmos
na era da pós-verdade, onde tudo pode ser questionar, saber se encontrar em meios aos
mares de dados é uma habilidade imprenscindível para alcançar a erudição nos dias de hoje.

    As TDICS se provaram a base da educação em tempos de pandemia, contudo em tempos normais elas
visam aprimorar o ensino, a possibilidade de estudar remotamente em tempos sem o isolamente social, permite
os educandos participarem de atividades extra curriculares muito além do seu alcance sem as tecnologias.

\begin{citacao}
Um  novo  ensino  superior  aberto,  híbrido,  disruptivo,  multimodal,
pervasivo  e  ubíquo voltado para o atendimento personalizado das demandas
formativas dos estudantes e que  seja  consoante  com  a  sociedade  atual.
Essas  características  norteiam  a  atual cultura digital e influenciam as
necessidades de mudanças e inovações nos sistemas de educação
universitária.(KENSKI,2019).
\end{citacao}

    Este aspecto multimodal revulucionou a forma de ensino, pois ao fornecer
uma visão muito mais imersiva e poderosa de ensino, os educandos podem adiquirir
informação e aprender culturas em uma magnitude nunca antes vista. O acesso à
produção científica e a literatura clássica também revolucionou a forma de fazer pesquisa, a experimentação
realizada pelos educandos com as novas tecnologias, e a informação no formato multimídia fez que aprender
de forma empírica se tornasse muito mais acessível.

    Com o sistema remoto, o conceito de presença e o tamanho do ambiente de ensino foram redesenhados, educar
não se limita mais a um espaço ou à escola.

\begin{citacao}
O ensino híbrido implica na integração entre ambientes de ensino superior presenciais e  online.  Institui-se  a  convergência  de  práticas  de  ensino  e  de  aprendizagem,  na configuração do blended learning(BL), que modifica o conceito de presença, tanto do professor  quanto  do  aluno.  A  aula  se  amplia  e  incorpora  o  melhor  de  dois  ou  mais ambientes –presenciais e virtuais. A redução de distâncias em ensino e aprendizagem fundamenta essa nova sala de aula,que pode transformar a universidade num lugar “sem distâncias”. E sem muros e barreiras.(KENSKI,2019)
\end{citacao}

\section{Como ensinar tecnologia nas escolas de maneira efetiva}

    Para realizar o ensino da tecnologia nas escolas é possível fazer dos conhecimentos
já do domínio do educando, onde ao estudar e abordar as questões que dizem a respeito
a tecnologia de seu dia a dia profundamente, é possível abstrair os processos computacionais
e fornecer uma visão diferente ao educando.

    Um exemplo seria o uso de navegadores de internet, estes navegadores permitem que o
usuário consiga interagir com o conteúdo e os mecanismos do navegador de uma maneira mais técnica
ao interagir com as ferramentas de desenvolvedor, ferramentas essa que possibilitam o ensino de uma linguagem
de programação de script, como funciona uma página web e conceitos de redes de computadores através de
dados de performance. O ensino mais elaborado pode ser realizado com planejamento de aulas utlizando
linguagens de programação para ensino, como é o caso do scratch ou portugol.

    O ensino mais avançado de conceitos computacionais pode ser realizado com o uso de sistemas
microcontroladores didáticos, como é o caso do arduino, uma placa que permite o usuário
programar um sistema embarcado de maneira mais simples, com esta placa é possível realizar
inúmeras de projetos com apelo ao lúdico e também é possível desenvolver projetos complexos
gradativamente, graças a extensa comunidade dos entusiastas do arduino, é possível encontrar
vários projetos já realizados para replicação e cocriar novos projetos com outras pessoas.

    Um artíficio frequemente usado por professores é incentivar o estudo de algoritmos através do
hacking de jogos, jogos populares de hoje permitem que o jogador interage com as partes internas
mais complexa do jogo, assim é possível aprender conceitos novos através da exploração desses mecanismos.

    Uma questão ainda alarmante na implantação do ensino da tecnologia é a carência do acesso da internet
por parte da população brasileira, isto dificulta que pessoas sejam integradas às novas culturas, acentuando
ainda mais desigualdades. Para essas pessoas afetadas, ainda sim é possível recorrer às bibliotecas municipais e
projetos realizados pelo estado.

\section{O ensino de tecnologia como ferramenta para a cidadania}

    É do conhecimento do brasileiro que a tecnologia esta presente no
ato mais fundamental do exercício da democracia, a urna eletrônica, uma
referência mundial nas democracias modernas. Hoje o cidadão exerce sua cidadania
através do meio digital de maneria expressiva, se mobiliza em protestos através
de redes sociais, se informa através de veículos de notícias e se envolve politicamente
através da internet, se mobiliza através das redes sociais, inúmeros processos políticos recentes
na história do Brasil se iniciaram nas redes sociais.

    Ainda assim o cidadão moderno não compreende a extenção de seus poderes, os novos sistemas
permitem ele fiscalizar os poderes da república, impactar diretamente no processo legislativo
através dos dados obtidos através dos portais do governo.

    Com a integração dos sistemas burocráticos aos sistemas informatizados, como obtenção de documentos
e o uso de serviços do governo, a integração digital se torna essencial ao cidadão brasileiro. Na era da informação, para ser cidadão, é preciso ser
digital.

    Na internet o debate sobre injustiças sociais ganha voz, e grupos periféricos podem
serem melhor integrados nos processos sociais, conseguem aceder educação e se politizar, onde
as estratégias do governo para combater a injustiça se provam ineficientes, o protagonismo do autodidatismo
toma espaço.


\section{Conclusão}
Aqui escrevo a conclusão


\postextual
% ----------------------------------------------------------
% Referências bibliográficas
% ----------------------------------------------------------
\bibliography{abntex2-modelo-references}
% ----------------------------------------------------------
% Agradecimentos
% ----------------------------------------------------------
\section*{Agradecimentos}
Texto sucinto aprovado pelo periódico em que será publicado. Último
elemento pós-textual.
\end{document}
